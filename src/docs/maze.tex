% Generated by GrindEQ Word-to-LaTeX

\documentclass[english,russian]{article} % use \documentstyle for old LaTeX compilers
\usepackage[utf8]{inputenc} % 'cp1252'-Western, 'cp1251'-Cyrillic, etc.
% \usepackage{newunicodechar}
% \newunicodechar{fi}{fi}
% \usepackage[english]{babel} % 'french', 'german', 'spanish', 'danish', etc.
\usepackage[english,russian]{babel}
\usepackage[T1,T2A,TS1]{fontenc}
\usepackage{amsmath}
\usepackage{amssymb}
\usepackage{txfonts}
% \usepackage{mathdots}
% \usepackage[classicReIm]{kpfonts}
\usepackage{graphicx}

\begin{document}

\title{Maze}

\date{\today}

\maketitle


\begin{enumerate}
\item Информация о Maze

В данном проекте Вам предстоит познакомиться с лабиринтами и пещерами, а также основными алгоритмами их обработки, такими как: генерация, отрисовка, поиск решения.

\item Реализация проекта Maze
\end{enumerate}
\begin{itemize}
\item Программа должна быть разработана на языке C++ стандарта C++17
\item Код программы должен находиться в папке src
\item При написании кода необходимо придерживаться Google Style
\item Сборка программы должна быть настроена с помощью Makefile со стандартным набором целей для GNU-программ: all, install, uninstall, clean, dvi, dist, tests. Установка должна вестись в любой другой произвольный каталог
\item В программе должен быть реализован графический пользовательский интерфейс на базе любой GUI-библиотеки с API для C++ (Qt, SFML, GTK+, Nanogui, Nngui, etc.)
\item В программе предусмотрена кнопка для загрузки лабиринта из файла, который задается в формате, описанном выше
\item Максимальный размер лабиринта - 50х50
\item Загруженный лабиринт должен быть отрисован на экране в поле размером 500 x 500 пикселей
\item Толщина "стены" - 2 пикселя
\item Размер самих ячеек лабиринта вычисляется таким образом, чтобы лабиринт занимал всё отведенное под него поле\end{itemize}
\end{document}
